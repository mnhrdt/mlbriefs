\documentclass[t]{beamer}
\usepackage[utf8]{inputenc}  % to be able to type unicode text directly
%\usepackage[french]{babel}   % french typographical conventions
\usepackage{inconsolata}     % for a nicer (e.g. non-courier) tt family font
%\usepackage{amsthm,amsmath}  % fancier mathematics
\usepackage{array} % to fine-tune tabular spacing
\usepackage{bbm} % for blackboard 1

\usepackage{graphicx}        % to include images
%\usepackage{animate}         % to include animated images
\usepackage{soul}            % for colored strikethrough
%\usepackage{bbding}          % for Checkmark and XSolidBrush
\usepackage{hyperref,url}

\colorlet{darkgreen}{black!50!green}  % used for page numbers
\definecolor{term}{rgb}{.9,.9,.9}     % used for code insets

\setlength{\parindent}{0em}
\setlength{\parskip}{1em}


% coco's macros
\def\R{\mathbf{R}}
\def\F{\mathcal{F}}
\def\x{\mathbf{x}}
\def\y{\mathbf{y}}
\def\u{\mathbf{u}}
\def\Z{\mathbf{Z}}
\def\d{\mathrm{d}}
\DeclareMathOperator*{\argmin}{arg\,min}
\DeclareMathOperator*{\argmax}{arg\,max}
\newcommand{\reference}[1] {{\scriptsize \color{gray}  #1 }}
\newcommand{\referencep}[1] {{\tiny \color{gray}  #1 }}
\newcommand{\unit}[1] {{\tiny \color{gray}  #1 }}

% disable spacing around verbatim
\usepackage{etoolbox}
\makeatletter\preto{\@verbatim}{\topsep=0pt \partopsep=0pt }\makeatother

% disable headings, set slide numbers in green
\mode<all>\setbeamertemplate{navigation symbols}{}
\defbeamertemplate*{footline}{pagecount}{\leavevmode\hfill\color{darkgreen}
   \insertframenumber{} / \inserttotalframenumber\hspace*{2ex}\vskip0pt}

%% select red color for strikethrough
\makeatletter
\newcommand\SoulColor{%
  \let\set@color\beamerorig@set@color
  \let\reset@color\beamerorig@reset@color}
\makeatother
\newcommand<>{\St}[1]{\only#2{\SoulColor\st{#1}}}
\setstcolor{red}

% make everything monospace
\renewcommand*\familydefault{\ttdefault}

% define a font size tinier than tiny
\makeatletter
\newcommand{\srcsize}{\@setfontsize{\srcsize}{5pt}{5pt}}
\makeatother


\begin{document}

\addtocounter{framenumber}{-1}
\begin{frame}[plain,fragile]
\LARGE\begin{verbatim}





   Kadkhodaie-Simoncelli dynamics




mnhrdt
gtti 1/2/2024
\end{verbatim}
\end{frame}

\begin{frame}
CONTEXT: GTTIS ABOUT DIFFUSION MODELS\\
=====================================

\vfill
\begin{columns}
	\begin{column}{0.3\textwidth}\srcsize
		{\bf Valentin de Bortoli},
		{\color{gray} October 2023}\\
		{Diffusion Schrödinger Bridge and Generative Modeling}
		\vspace{4em}

		\includegraphics[width=\linewidth]{f/borto_shot1.png}
		\vspace{4em}

		\includegraphics[width=\linewidth]{f/borto_shot2.png}
	\end{column}
	\pause
	\begin{column}{0.3\textwidth}\srcsize
		{\bf Zhe Zheng},
		{\color{gray} January 2024}\\
		{Denoiser and Its Application Beyond Denoising}
		\vspace{1em}

		\includegraphics[width=\linewidth]{f/zhe_shot0.png}
		\vspace{1em}

		\includegraphics[width=\linewidth]{f/zhe_shot1.png}
		\vspace{1em}

		\includegraphics[width=\linewidth]{f/zhe_shot2.jpg}
	\end{column}
	\pause
	\begin{column}{0.3\textwidth}\srcsize
		{\bf Enric Meinhardt-Llopis},
		{\color{gray} February 2024}\\
		\vspace{4em}

		\pause
		\includegraphics[width=\linewidth]{f/compleat.jpg}
	\end{column}
\end{columns}
\vfill
\end{frame}

\begin{frame}
CONTEXT: THE BASIC IDEA (1)\\
===========================

\vfill
\includegraphics[width=\linewidth]{f/gobrrr.png}
\vfill
\end{frame}




\end{document}

%\begin{frame}
%IMPLICIT IMAGE MODELS OF CLASSICAL DENOISERS\hfill{\footnotesize{\color{gray}mnhrdt}}\\
%============================================
%
%\small
%Any denoiser contains an ``implicit image prior''.\\
%How can we can extract images from it?
%
%{\color{gray}Kadkhodaie-Simoncelli 2021}\\
%\includegraphics[width=0.7\linewidth]{ks.png}
%
%%\vfill
%{\bf Demo: } Bring your own denoiser for this algorithm
%
%%\vfill
%\tiny
%\begin{tabular}{lll}
%	\includegraphics[height=0.18\textheight]{somiters3b.png} &
%	\includegraphics[height=0.18\textheight]{songermon1b.png} &
%	\includegraphics[height=0.18\textheight]{songermon1a.png} \\
%	Denoiser = median &
%	Denoiser = DCNN (generic) &
%	Denoiser = DCNN (faces)
%\end{tabular}
%
%\end{frame}
%
%\begin{frame}
%IMPLICIT IMAGE MODELS OF CLASSICAL DENOISERS
%============================================
%
%%Computed with current code:
%%Particular cases of Kadkhodaie-Simoncelli dynamics:
%
%\tiny
%\begin{tabular}{lll}
%	\includegraphics[width=0.31\textwidth]{out_perlin.png} &
%	\includegraphics[width=0.31\textwidth]{out_mcm.png} &
%	\includegraphics[width=0.31\textwidth]{out_dct.png} \\
%	gausssian blur (perlin noise) &
%	median filter (mcm) &
%	dct denoising \\
%	&&\\
%	\includegraphics[width=0.31\textwidth]{out_nlm0.png} &
%	\includegraphics[width=0.31\textwidth]{out_nlbayes.png} &
%	\includegraphics[width=0.31\textwidth]{out_ffdnet.png} \\
%	non-local means (nl diffusion) &
%	non-local bayes &
%	ffdnet \\
%\end{tabular}
%
%\end{frame}
%
%\begin{frame}
%SUBPRODUCT: PYTHON INTERFACE TO IPOL ALGORITHMS
%===============================================
%
%\includegraphics[height=0.9\textheight]{clipolcollab.png}
%\end{frame}
%
%
%\end{document}


% vim:sw=2 ts=2 spell spelllang=fr:
