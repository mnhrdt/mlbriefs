\documentclass[t]{beamer}
\usepackage[utf8]{inputenc}  % to be able to type unicode text directly
%\usepackage[french]{babel}   % french typographical conventions
\usepackage{inconsolata}     % for a nicer (e.g. non-courier) tt family font
%\usepackage{amsthm,amsmath}  % fancier mathematics
\usepackage{array} % to fine-tune tabular spacing
\usepackage{bbm} % for blackboard 1

\usepackage{graphicx}        % to include images
%\usepackage{animate}         % to include animated images
\usepackage{soul}            % for colored strikethrough
%\usepackage{bbding}          % for Checkmark and XSolidBrush
\usepackage{hyperref,url}

\colorlet{darkgreen}{black!50!green}  % used for page numbers
\definecolor{term}{rgb}{.9,.9,.9}     % used for code insets

\setlength{\parindent}{0em}
\setlength{\parskip}{1em}


% coco's macros
\def\R{\mathbf{R}}
\def\F{\mathcal{F}}
\def\x{\mathbf{x}}
\def\y{\mathbf{y}}
\def\u{\mathbf{u}}
\def\Z{\mathbf{Z}}
\def\d{\mathrm{d}}
\DeclareMathOperator*{\argmin}{arg\,min}
\DeclareMathOperator*{\argmax}{arg\,max}
\newcommand{\reference}[1] {{\scriptsize \color{gray}  #1 }}
\newcommand{\referencep}[1] {{\tiny \color{gray}  #1 }}
\newcommand{\unit}[1] {{\tiny \color{gray}  #1 }}

% disable spacing around verbatim
\usepackage{etoolbox}
\makeatletter\preto{\@verbatim}{\topsep=0pt \partopsep=0pt }\makeatother

% disable headings, set slide numbers in green
\mode<all>\setbeamertemplate{navigation symbols}{}
\defbeamertemplate*{footline}{pagecount}{\leavevmode\hfill\color{darkgreen}
   \insertframenumber{} / \inserttotalframenumber\hspace*{2ex}\vskip0pt}

%% select red color for strikethrough
\makeatletter
\newcommand\SoulColor{%
  \let\set@color\beamerorig@set@color
  \let\reset@color\beamerorig@reset@color}
\makeatother
\newcommand<>{\St}[1]{\only#2{\SoulColor\st{#1}}}
\setstcolor{red}

% make everything monospace
\renewcommand*\familydefault{\ttdefault}

\begin{document}


\begin{frame}[fragile]
VISUALIZATION OF TOPOGRAPHIC MAPS\\
=================================


%lluc meinhardt

\tiny
\begin{verbatim}
0. History, context
0.1. What is a TM, DEM, DSM, DTM
0.2. History of map colorization
0.3. Fuji 3d photos, identify landmarks
0.4. Rotate 3d model interactively

1. Symbolic-based methods
1.1. Palette (gray, color)
1.2. Level lines
1.3. Level lines + palette (zelda!)
1.4. Slope = closeness of level lines, norm of gradient
1.5. Slope angle = direction of level lines, gradient vector field, grad.-form
1.6. Curvature sign = convexity, concavity of level lines

2. Light-based methods
2.1. Shadows
2.2. Lambertian shading (gouraud, + linear approximation)
2.3. Clear sky shading (ao sampling, + linear approximation)
2.4. Specular shading (phong)
2.5. SAR direct model

3. Uses and fancy things
3.1. To observe defects, understand geometry, etc
3.2. Combined methods: tinted hillshade with level-lines
3.3. As a texture for a 3D rendering
3.4. As a direct model for several famous inverse problems
3.5. Take home message: do not show "north up" in optical images
3.6. Colophon: this document is a notebook
\end{verbatim}

\end{frame}

\begin{frame}
HISTORY, CONTEXT\\
================
\end{frame}

\begin{frame}
What is a TM, DEM, DSM, DTM\\
===========================
\end{frame}

\begin{frame}
History of map colorization\\
================
\end{frame}

\begin{frame}
Fuji 3d photos, identify landmarks\\
================
\end{frame}

\begin{frame}
Rotate 3d model interactively\\
================
\end{frame}

\begin{frame}
Symbolic-based methods\\
================
\end{frame}

\begin{frame}
Palette (gray, color)\\
================
\end{frame}

\begin{frame}
Level lines\\
================
\end{frame}

\begin{frame}
Level lines + palette (zelda!)\\
================
\end{frame}

\begin{frame}
Slope = closeness of level lines, norm of gradient\\
================
\end{frame}

\begin{frame}
Slope angle = direction of level lines, gradient vector field, grad.-form\\
================
\end{frame}

\begin{frame}
Curvature sign = convexity, concavity of level lines\\

================
\end{frame}

\begin{frame}
Light-based methods\\
================
\end{frame}

\begin{frame}
Shadows\\
================
\end{frame}

\begin{frame}
Lambertian shading (gouraud, + linear approximation)\\
================
\end{frame}

\begin{frame}
Clear sky shading (ao sampling, + linear approximation)\\
================
\end{frame}

\begin{frame}
Specular shading (phong)\\
================
\end{frame}

\begin{frame}
SAR direct model\\
================
\end{frame}

\begin{frame}
Uses and fancy things\\
================
\end{frame}

\begin{frame}
To observe defects, understand geometry, etc\\
================
\end{frame}

\begin{frame}
Combined methods: tinted hillshade with level-lines\\
================
\end{frame}

\begin{frame}
As a direct model for several famous inverse problems\\
================
\end{frame}

\begin{frame}
Take home message: do not show "north up" in optical images\\
================
\end{frame}


\begin{frame}
COLOPHON\\
========

this document is a notebook
\end{frame}




\end{document}


% vim:sw=2 ts=2 spell spelllang=en:
